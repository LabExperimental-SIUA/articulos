% plantilla de IEEEtrans (usada por el clei)
% #apt-get install texlive-publishers
\documentclass[conference]{IEEEtran}

% *** GRAPHICS RELATED PACKAGES ***
\usepackage{graphicx}
% declare the path(s) where your graphic files are
\graphicspath{{imagenes/}}
% and their extensions so you won't have to specify these with
% every instance of \includegraphics
\DeclareGraphicsExtensions{.png}

%para usar acentos del español en el código fuente
% #apt-get install texlive-lang-spanish
\usepackage[spanish]{babel}
\selectlanguage{spanish}
\usepackage[utf8]{inputenc}

% correct bad hyphenation here
%\hyphenation{op-tical net-works semi-conduc-tor}


\begin{document}

% can use linebreaks \\ within to get better formatting as desired
\title{Laboratorio Experimental de Tecnologías Computacionales}

% author names and affiliations
\author{\IEEEauthorblockN{Aurelio Sanabria Rodríguez}
\IEEEauthorblockA{Instituto Tecnológico de Costa Rica\\
Sede Interuniversitaria de Alajuela\\
Alajuea, Costa Rica\\
ausanabria@itcr.ac.cr}
\and
\IEEEauthorblockN{Diego Munguía Molina}
\IEEEauthorblockA{Instituto Tecnológico de Costa Rica\\
Sede Interuniversitaria de Alajuela\\
Alajuea, Costa Rica\\
dmunguia@itcr.ac.cr}
\and
\IEEEauthorblockN{Jaime Gutiérrez Alfaro}
\IEEEauthorblockA{Instituto Tecnológico de Costa Rica\\
Sede Interuniversitaria de Alajuela\\
Alajuela, Costa Rica\\
jgutierrez@itcr.ac.cr}}

% make the title area
\maketitle




\begin{abstract}


%\boldmath
...Pendiente de ser hecho...
\end{abstract}

% no keywords
Palabras clave — innovación, creación, social, extensión
%%%%  Nota: revisar estas palabras claves



% For peer review papers, you can put extra information on the cover
% page as needed:
% \ifCLASSOPTIONpeerreview
% \begin{center} \bfseries EDICS Category: 3-BBND \end{center}
% \fi
%
% For peerreview papers, this IEEEtran command inserts a page break and
% creates the second title. It will be ignored for other modes.
\IEEEpeerreviewmaketitle



\section{Introducción}


Responde a esto: ¿cómo se estableció el espacio? y cómo fue en términos generales estableciéndose.

El laboratorio experimental de Tecnologías Computacionales (LabExp) es una actividad de investigación aprobada por el Comité Técnico del Centro de Investigaciones en Computación, está situado en la Sede Interuniversitaria de Alajuela[nota al pie: explicar que es la SIU y la participación de otras universidades] específicamente en el Instituto Tecnológico de Costa Rica. 

nota: Resumir en un párrafo la propuesta que se presentó al CIC.


\section{Esquemas de trabajo: general y específicos}

¿Cómo hemos estado trabajando?

\begin{itemize}

\item General
    
    \begin{itemize}
    
    \item Administrativa: tiempo de los profesores( adhonorem, con tiempo asignado), tiempo de los estudiantes (adhonorem, con horas beca (mecanismos TEC) o con becas específicas (comida y transporte --utilizado en los proyectos de verano--)
    
    \item Coordinación del trabajo del laboratorio: reuniones con específicas de proyectos, reuniones generales, interacción con la escuela de computación.
    
    \end{itemize}

\item Específicas: ¿cómo se trabaja en cada proyecto en cada momento del tiempo?
    
    \begin{itemize}
    \item Escáner (¿?... explicar la modalidad tomando en cuenta que se tenía que trabajar la propuesta del CIC y las tareas del laboratorio)
    
    \item Buses (¿datos abiertos?... explicar la modalidad tomando en cuenta que se tenía que trabajar la propuesta del CIC y las tareas del laboratorio)
    
    \item Relojeros
    
    \item Proyectos de Verano

    \end{itemize}


\end{itemize}
  
Nota: Tomar en cuentas aspectos técnicos y de divulgación (wiki, issues, código, posts del blog).  reuniones, asignación de trabajo, expectativas iniciales de trabajo, evaluación del trabajo y ajustes en la siguiente "iteración". 


% An example of a floating figure using the graphicx package.
% Note that \label must occur AFTER (or within) \caption.
% For figures, \caption should occur after the \includegraphics.
\begin{figure}[!t]
\centering
\includegraphics[width=2.5in]{escaner-estudiantes_trabajando}
% where an .eps filename suffix will be assumed under latex, 
% and a .pdf suffix will be assumed for pdflatex; or what has been declared
% via \DeclareGraphicsExtensions.
\caption{Estudiantes trabajando en el ensamblaje del escaner.}
\label{escaner}
\end{figure}
%%Ayuda: Falta una tile en escáner pero tex croma cuando la agrego

% Note that IEEE typically puts floats only at the top, even when this
% results in a large percentage of a column being occupied by floats.



\section{Resultados}


Al finalizar el 2014: 2 borradores de propuestas de investigación para ser presentadas a las VIE en la ronda del 2015, dado que la presentación de las propuestas es en mayo todavía se dispone de bastante tiempo para refinarlas. 

2013-2014:Exposición a público general: días de la libertad de software (Alajuela, limón)[nota al pie: explicar qué es el SFD, que fue apoyado por la municipalidad de Alajuela y que tuvo amplia difusión en medios de prensa], semana Interunivesitaria (buses), día de la sede. 

Al finalizar 2014: Espacio físico destinado para trabajar y materiales suficientes para cumplir con el propósito de experimentar con tecnologías.  



% An example of a floating figure using the graphicx package.
% Note that \label must occur AFTER (or within) \caption.
% For figures, \caption should occur after the \includegraphics.
\begin{figure}[!t]
\centering
\includegraphics[width=2.5in]{escaner-mitad_ensamblaje}
% where an .eps filename suffix will be assumed under latex, 
% and a .pdf suffix will be assumed for pdflatex; or what has been declared
% via \DeclareGraphicsExtensions.
\caption{El escaner a mitad del ensamblaje.}
\label{escaner}
\end{figure}
%%Ayuda: Falta una tile en escáner pero tex croma cuando la agrego

% Note that IEEE typically puts floats only at the top, even when this
% results in a large percentage of a column being occupied by floats.




\section{Conclusiones y trabajo futuro}

Al finalizar el 2014 nos parece que el LabExp ha ofrecido a los estudiantes involucrados un espacio donde:

\begin{itemize}

\item se enfrentaron a problemas técnicos novedosos para ellos, 

\item profundiza en conceptos de Ingeniería de Software

\item Software libre -> ver código

\item ambientes de programación grupales (habilidades sociales de programación)

\item un proceso de investigación con los estudiantes.

\end{itemize}

Trabajo futuro: consolidación de los proyectos y esquemas de trabajo, mejorar la divulgación de los proyectos (principalmente en el ámbito académico), establecer mecanismos para incorporar nuevos miembros (profesores y estudiantes), establecer vínculos con otras carreras de la SIU para trabajar en conjunto. 


% references section

% can use a bibliography generated by BibTeX as a .bbl file
% BibTeX documentation can be easily obtained at:
% http://www.ctan.org/tex-archive/biblio/bibtex/contrib/doc/
% The IEEEtran BibTeX style support page is at:
% http://www.michaelshell.org/tex/ieeetran/bibtex/
\bibliographystyle{IEEEtran}
% argument is your BibTeX string definitions and bibliography database(s)
\bibliography{labexp1.bib}
%
% <OR> manually copy in the resultant .bbl file
% set second argument of \begin to the number of references
% (used to reserve space for the reference number labels box)
%\begin{thebibliography}{3}


%\end{thebibliography}




% that's all folks
\end{document}



%\subsection{Subsection Heading Here}
%Subsection text here.


%\subsubsection{Subsubsection Heading Here}
%Subsubsection text here.


% An example of a floating figure using the graphicx package.
% Note that \label must occur AFTER (or within) \caption.
% For figures, \caption should occur after the \includegraphics.
% Note that IEEEtran v1.7 and later has special internal code that
% is designed to preserve the operation of \label within \caption
% even when the captionsoff option is in effect. However, because
% of issues like this, it may be the safest practice to put all your
% \label just after \caption rather than within \caption{}.
%
% Reminder: the "draftcls" or "draftclsnofoot", not "draft", class
% option should be used if it is desired that the figures are to be
% displayed while in draft mode.
%
%\begin{figure}[!t]
%\centering
%\includegraphics[width=2.5in]{myfigure}
% where an .eps filename suffix will be assumed under latex, 
% and a .pdf suffix will be assumed for pdflatex; or what has been declared
% via \DeclareGraphicsExtensions.
%\caption{Simulation Results}
%\label{fig_sim}
%\end{figure}

% Note that IEEE typically puts floats only at the top, even when this
% results in a large percentage of a column being occupied by floats.


% An example of a double column floating figure using two subfigures.
% (The subfig.sty package must be loaded for this to work.)
% The subfigure \label commands are set within each subfloat command, the
% \label for the overall figure must come after \caption.
% \hfil must be used as a separator to get equal spacing.
% The subfigure.sty package works much the same way, except \subfigure is
% used instead of \subfloat.
%
%\begin{figure*}[!t]
%\centerline{\subfloat[Case I]\includegraphics[width=2.5in]{subfigcase1}%
%\label{fig_first_case}}
%\hfil
%\subfloat[Case II]{\includegraphics[width=2.5in]{subfigcase2}%
%\label{fig_second_case}}}
%\caption{Simulation results}
%\label{fig_sim}
%\end{figure*}
%
% Note that often IEEE papers with subfigures do not employ subfigure
% captions (using the optional argument to \subfloat), but instead will
% reference/describe all of them (a), (b), etc., within the main caption.


% An example of a floating table. Note that, for IEEE style tables, the 
% \caption command should come BEFORE the table. Table text will default to
% \footnotesize as IEEE normally uses this smaller font for tables.
% The \label must come after \caption as always.
%
%\begin{table}[!t]
%% increase table row spacing, adjust to taste
%\renewcommand{\arraystretch}{1.3}
% if using array.sty, it might be a good idea to tweak the value of
% \extrarowheight as needed to properly center the text within the cells
%\caption{An Example of a Table}
%\label{table_example}
%\centering
%% Some packages, such as MDW tools, offer better commands for making tables
%% than the plain LaTeX2e tabular which is used here.
%\begin{tabular}{|c||c|}
%\hline
%One & Two\\
%\hline
%Three & Four\\
%\hline
%\end{tabular}
%\end{table}


% Note that IEEE does not put floats in the very first column - or typically
% anywhere on the first page for that matter. Also, in-text middle ("here")
% positioning is not used. Most IEEE journals/conferences use top floats
% exclusively. Note that, LaTeX2e, unlike IEEE journals/conferences, places
% footnotes above bottom floats. This can be corrected via the \fnbelowfloat
% command of the stfloats package.


% trigger a \newpage just before the given reference
% number - used to balance the columns on the last page
% adjust value as needed - may need to be readjusted if
% the document is modified later
%\IEEEtriggeratref{8}
% The "triggered" command can be changed if desired:
%\IEEEtriggercmd{\enlargethispage{-5in}}
